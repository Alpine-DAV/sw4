%-*-LaTeX-*-
\documentclass[11pt]{report}
%%%%%%%%%%%%%%%%%%%%%%%%%%%%%%%%%%%%%%%%
% Use the approved methods for setting lengths
%\oddsidemargin  0.0in
%\evensidemargin 0.0in
%\textwidth      6.5in
%\textheight     9.0in
\setlength{\oddsidemargin}{0.0in}
\setlength{\evensidemargin}{0.0in}
\setlength{\textwidth}{6.5in}
\setlength{\textheight}{8.7in}
% Based on document style, and taller text body height, set
% weird LaTex margin/header (box heights).  This fixes
% the disappearing page numbers (which never disappeared; they
% just got printed beyond the borders of the physical paper)
\setlength{\voffset}{0pt}
\setlength{\topmargin}{0pt}
\setlength{\headheight}{10pt}
\setlength{\headsep}{25pt}
%%%%%%%%%%%%%%%%%%%%%%%%%%%%%%%%%%%%%%%%
%\usepackage{html}
\usepackage{makeidx}
%\usepackage{times}
\usepackage{hyperref}
\usepackage{verbatim}
\usepackage{graphicx}
%\usepackage{eufrak}
\usepackage{amsmath}

\makeindex

\tolerance=600

\newcommand{\Fb}{{\bf F}}
\newcommand{\Lb}{{\bf L}}
\newcommand{\gb}{{\bf g}}
\newcommand{\nb}{{\bf n}}
\newcommand{\ub}{{\bf u}}
\newcommand{\xb}{{\bf x}}
\newcommand{\p}{\partial}
\renewcommand{\div}{{\rm div}}
\renewcommand{\arraystretch}{1.3}

\begin{document}

\title{\LARGE User's guide to SW4Opt version 0.0.1} 

\author{ N. Anders Petersson$^1$ \and Bj\"orn Sj\"ogreen\thanks{Center for Applied Scientific
     Computing, Lawrence Livermore National Laboratory, PO Box 808, Livermore CA 94551. This is
     contribution LLNL-SM-XXXYYY.}}
\date{\today} 
\maketitle
\pagestyle{myheadings}

%\thispagestyle{plain}
\markboth{}{N.A. PETERSSON AND B. SJOGREEN; USER'S GUIDE TO SW4OPT-0.0.1}

\pagebreak
\paragraph {Disclaimer} 
This document was prepared as an account of work sponsored by an agency of the United States
government. Neither the United States government nor Lawrence Livermore National Security, LLC, nor
any of their employees makes any warranty, expressed or implied, or assumes any legal liability or
responsibility for the accuracy, completeness, or usefulness of any information, apparatus, product,
or process disclosed, or represents that its use would not infringe privately owned
rights. Reference herein to any specific commercial product, process, or service by trade name,
trademark, manufacturer, or otherwise does not necessarily constitute or imply its endorsement,
recommendation, or favoring by the United States government or Lawrence Livermore National Security,
LLC. The views and opinions of authors expressed herein do not necessarily state or reflect those of
the United States government or Lawrence Livermore National Security, LLC, and shall not be used for
advertising or product endorsement purposes. 

\paragraph{Auspices Statement}
This work performed under the auspices of the U.S. Department of Energy by Lawrence Livermore
National Laboratory under contract DE-AC52-07NA27344.
%\pagebreak
\tableofcontents

% Introduction
\chapter{Source inversion}
\emph{S4Wopt} is the main program for source inversion. \emph{S4Wopt} estimates a source of the form
$$
 f({\bf x},t) =  g(t,t_0,\omega_0)M\nabla\delta({\bf x}-{\bf x}^*)
$$
where, ${\bf x}^*$ is the location of the source, $M$ is the symmetric 3$\times$3 moment tensor, and the time
function $g$ contains two parameters, the frequency, $\omega_0$, and time shift, $t_0$. Currently, the only
implemented time functions for source inversion are the Gaussian and the VerySmoothBump.
\par
The source is computed by minimizing the misfit between the computed
time series ${\bf u}^n_{{\bf i}_r}$ and measured data ${\bf d}^n_{{\bf i}_r}$,
$$
 \chi = \frac{1}{2}\sum_{n=0}^N \sum_{r=1}^R |{\bf u}^n_{{\bf i}_r}-{\bf d}^n_{{\bf i}_r}|^2,
$$
with respect to the eleven parameters describing the source, ${\bf x}^*, M, \omega_0$, and $t_0$.
The measured data are given at stations ${\bf x}_{{\bf i}_r}$, $r=1,\ldots,R$.
The {\tt observation} keyword is used to specify the input seismograms (${\bf d}^n_{{\bf i}_r}$ above). 
For example, the command
%
\begin{verbatim}
observation x=11200 y=6300 z=0.0 file=cartstation
\end{verbatim}
%
will read a file named {\tt cartstation.txt} from the directory specified by the {\tt path}
parameter to the {\tt fileio} command. If the {\tt path} is not set, the file will be searched for
in the current directory. Currently, only files on the USGS ascii format can be read by the
observation command. The USGS file do not contain information about the location of the station
where the data were obtained. It is thus necessary to specify the location directly in the {\tt
  observation} command. In the example above, the data on the file represent measurements at a
station located at the position ${\bf x} = (11200,6300,0)$ in the Cartesian grid. It is possible to
specify the location in geographic coordinates instead. For example,
%
\begin{verbatim}
observation lon=-122.523 lat=38.0496 depth=0.0 file=nhf 
\end{verbatim}
%
will read a file named {\tt nhf.txt}, that represents data obtained at latitude 38.049 and longitude
-122.523 on the surface of the earth.  \emph{S4Wopt} will interpret the components of read file as
Cartesian components, or as components pointing in the east, north, and up directions, depending on
information in the header of the file.  

The file read by the observation command will be
assumed to start at time = 0, even if the starting time recorded in the file is different from zero,
i.e., the read seismograms will be shifted in time to always start at the first time step of the
simulation.

%\begin{narrow}
{\emph Not Yet Implemented:}
Plan for reading sac files:
\begin{verbatim}
observation sfile1=filename.x sfile2=filename.y sfile3=filename.z
\end{verbatim}
will read the three components of a time history at a point from three SAC files. Information about type of
components and location of the station is taken from the header of the SAC file.
%\end{narrow}

The stations specified by the observation command, will also act as recievers of time series during
the forward solve phase. After each iteration in the optimization algorithm, the time histories
obtained at the stations (${\bf u}^n_{{\bf i}_r}$ above) will be output with on files with a {\tt
  \_out} appended to their names. For example, if time history is input on the file {\tt nhf.txt},
the corresponding seismograms computed by the forward problem using the most recent approximation of
the source are written on a file named {\tt nhf\_out.txt}. These files are thus overwitten after each
optimization iteration. Furthermore, if the option output\_initial\_seismograms is set to one, the
time series obtained with the initial guess source will be output on files with a {\tt \_ini}
appended.  

The {\tt receiver} command, defined for \emph{S4W}, can not be used with
\emph{S4Wopt}. Currently, it is not possible to specify a receiver that is not also an observation.

%\chapter{Introduction}
%\chapter{Getting started}
\chapter{Coordinate system, units and the grid}

The computation takes place over a time interval, $0\leq t \leq T$, where $T$ is a user specified end time.
The starting point of the simulation time is always $t=0$. The {\tt time} command specifies the end time, for example
\begin{verbatim}
time t=1.6
\end{verbatim}
sets $T=1.6$. The unit is seconds. Alternatively, the simulation time interval can be specified as 
a number of time steps, as in the example
\begin{verbatim}
time steps=1200
\end{verbatim}
which sets the number of time steps to 1200. The end time will in this case be $T=1200\Delta t$, where the
time step $\Delta t$ depends on the input material. $\Delta t$ is determined automatically by \emph{sw4}.
The simulation start time can be related to a UTC time. The option {\tt utcstart} sets the UTC time that
corresponds to the simulation time $t=0$, for example,
\begin{verbatim}
time t=1.6 utcstart=01/31/2012:17:34:12.233
\end{verbatim}
The format of the UTC time is ``month/day/year:hour:minute:second.milisecond''. When the UTC time is set,
output receiver data files are time stamped with the UTC reference time. Furthermore, the UTC start time
is essential to align UTC stamped observed data correctly when solving the inverse problem by \emph{sw4opt}.


%\chapter{Sources, time-functions and grid sizes}
%\chapter{The material model}
%\chapter{Topography} \label{sec:topography}
%\chapter{Mesh refinement} \label{sec:mesh-ref}
%\chapter{Attenuation}\label{sec:attenuation}
%\chapter{Output options}
%\chapter{Examples} \label{sec:examples}
%\chapter{Keywords in the input file}
%\chapter{File formats}
%\chapter{Installing \emph{WPP}}\label{cha:installing-wpp}
%\chapter{Testing the \emph{WPP} installation}

\chapter{Keywords in the input file}
%%%%%%%%%%%%%%%%%%%%%%%%%%%%%%%%%%%%%%%%%%%%%%%%
The syntax of the input file is
\begin{verbatim}
command1 parameter1=value1 parameter2=value2 ... parameterN=valueN
# comments are disregarded
command2 parameter1=value1 parameter2=value2 ... parameterN=valueN
...
\end{verbatim}
Each command starts at the beginning of the line and ends at the end of the same line. Blank and
comment lines are disregarded. A comment is a line starting with a \# character. The order of the
parameters within each command is arbitrary. The material commands (block, ifile, pfile, and efile)
are applied in the order they appear, but the ordering of all other commands is
inconsequential. Also note that the entire input file is read before the simulation starts.

Parameter values are either integers (-2,0,5,...), floating point numbers (20.5, -0.05, 3.4e4), or strings
(wpp, earthquake, my-favorite-simulation). Note that there must be no spaces around the = signs and
strings are given without quotation marks and must not contain spaces. Depending on the specific
command, some parameter values are required to fall within specified ranges.

A breif description of all commands is given in the following sections. The commands marked as
[required] must be present in all S4W input files, while those marked as [optional] are just
that. Other commands, such as those specifying the material model can be given by (a combination of)
different commands (block, pfile, efile, or ifile). Unless \emph{S4W} is run in one of its test
modes, the material must be specifed by at least one of these commands and at least one source must
be specified.

\section{Basic commands}

\section{Commands for source inversion}

\subsection{observation [required]}
\index{command!observation}
%%%%%%%%%%%%%%%%%%%%%%%%%%%%%%%%%%
The \verb+observation+ command is used to read an observed time history at a fixed
location in space. 
\begin{flushleft}
\bf
Syntax:\\
\tt
observation x=... y=... z=... lat=... lon=... depth=... file=... shift=... utc=... windowL=... windowR=... exclude=... filter=... sacfile1=... sacfile2=... sacfile3=... 
\\
{\bf Required parameters:}\\
\rm Location of the observation in Cartesian or geographical coordinates, and data file name(s).
The input data must be given, either as one usgs formated data file or as three SAC formated data files.
\end{flushleft}
%
\begin{center}
\begin{tabular}{|l|p{8cm}|l|l|l|} \hline
\multicolumn{5}{|c|}{\bf observation command parameters (part 1)}\\ \hline
\bf{Option} & \bf{Description} & \bf{Type} & \bf{Units} & \bf{Default} \\ \hline \hline
x & x position of the observation & float & m & none \\ \hline
y & y position of the observation & float & m & none \\ \hline
z & z position of the observation & float & m & none \\ \hline
\hline
lat & latitude geographical coordinate of the observation & float & degrees & none \\ \hline
lon & longitude geographical coordinate of the observation & float & degrees & none \\ \hline
depth & depth of the observation (below topography) & float & m & none \\ \hline
\hline
file & file name, usgs formated data  & string & none & none \\ \hline
\hline
sacfile1 & file name, SAC formated data & string & none & none \\ \hline
sacfile2 & file name, SAC formated data & string & none & none \\ \hline
sacfile3 & file name, SAC formated data & string & none & none \\ \hline
\end{tabular}
\end{center}
\index{observation parameters!location - x, y, z, lat, lon, depth, file, sacfile1, sacfile2, sacfile3 } 


{\bf Optional parameters:}\\
%
\begin{center}
\begin{tabular}{|l|p{8cm}|l|l|l|} \hline
\multicolumn{5}{|c|}{\bf observation command parameters (part 2)}\\ \hline
\bf{Option} & \bf{Description} & \bf{Type} & \bf{Units} & \bf{Default} \\ \hline \hline
shift & additonal data offset & float & s  & 0 \\ \hline
utc & Reference UTC time coordinate for station & string & none & read from file \\ \hline
windowL & Left limit of window in misfit & float & s & $-\infty$  \\ \hline
windowR & Right limit of window in misfit & float & s & $\infty$  \\ \hline
exclude & Components to exclude from misfit & string & none & none  \\ \hline
filter & filter data, when a filter is given & string & none & yes \\ \hline
\end{tabular}
\end{center}
The value of the \verb+exclude+ option is a string combining components 'x','y','z' or 'n','e','u'.

\subsection{scalefactors}
\index{command!scalefactors}
%%%%%%%%%%%%%%%%%%%%%%%%%%%%%%%%%%
The \verb+scalefactors+ command is used to define scale factors for improved conditioning of the
minimization problem.
\begin{flushleft}
\bf
Syntax:\\
\tt
scalefactors x0=... y0=... z0=... Mxx=... Mxy=... Mxz=... Myy=... Myz=... Mzz=... t0=... freq=... 
\\
\end{flushleft}
%
\begin{center}
\begin{tabular}{|l|p{8cm}|l|l|l|} \hline
\multicolumn{5}{|c|}{\bf scalefactors command parameters }\\ \hline
\bf{Option} & \bf{Description} & \bf{Type} & \bf{Units} & \bf{Default} \\ \hline \hline
x0 & scale of x position of source & float & m & 1 \\ \hline
y0 & scale of y position of source & float & m & 1 \\ \hline
z0 & scale of z position of source & float & m & 1 \\ \hline
Mxx & scale of xx moment tensor component & float & N m  & 1 \\ \hline
Mxy & scale of xy moment tensor component & float & N m  & 1 \\ \hline
Mxz & scale of xz moment tensor component & float & N m  & 1 \\ \hline
Myy & scale of yy moment tensor component & float & N m  & 1 \\ \hline
Myz & scale of yz moment tensor component & float & N m  & 1 \\ \hline
Mzz & scale of zz moment tensor component & float & N m  & 1 \\ \hline
t0 & scale of source time function shift & float & s & 1 \\ \hline
freq & scale of source time function frequency & float & s$^{-1}$ & 1 \\ \hline
\end{tabular}
\end{center}


\subsection{cg [required]}
\index{command!cg}
%%%%%%%%%%%%%%%%%%%%%%%%%%%%%%%%%%
The \verb+cg+ command defines parameters governing the conjugate gradient method
to solve the minimization problem.
\begin{flushleft}
\bf
Syntax:\\
\tt
cg maxit=... maxrestart=... tolerance=... initialguess=... write\_initial\_ts=... scalefactors=... linesearch=... steptype=... cgtype=... solvefor=...
\\
\end{flushleft}
%
\begin{center}
\begin{tabular}{|l|p{8cm}|l|l|l|} \hline
\multicolumn{5}{|c|}{\bf cg command parameters }\\ \hline
\bf{Option} & \bf{Description} & \bf{Type} & \bf{Units} & \bf{Default} \\ \hline \hline
maxit & maximum number of cg-iterations between restarts & int & none & 11 \\ \hline
maxrestart & maximum number of cg-restarts & int & none & 0 \\ \hline
tolerance & termination criterion on the scaled gradient & float & none & $10^{-6}$ \\ \hline
initialguess & method for determining the initial guess & string & none & usesource \\ \hline
write\_initial\_ts & output initial guess seismograms & boolean & none & 0 \\ \hline
scalefactors & method for determining scale factors & string & none & useinput \\ \hline
linesearch & switch on or off line search in cg minimization & string & none & on \\ \hline
steptype & method for computing initial step length for linesearch & string & none & misfit \\ \hline
cgtype & select variant of non-linear cg & string & none & fletcher-reeves \\ \hline
solvefor & solve for a subset of the 11 source parameters & string & none & posMt0freq \\ \hline
\end{tabular}
\end{center}

Possible values for some of the string valued options in the {\tt cg} command.
\begin{center}
\begin{tabular}{|l|p{8cm}|} \hline
\multicolumn{2}{|c|}{\bf cg command parameters }\\ \hline
\bf{Option} & \bf{Possible values}  \\ \hline \hline
initialguess & usesource, estimate, estimatePos, estimateT0Pos, estimateM \\ \hline
scalefactors & useinput, estimate  \\ \hline
steptype & misfit, hessian  \\ \hline
cgtype & fletcher-reeves, picola-ribiere \\ \hline
solvefor & posMt0freq, posMt0, posM \\ \hline
\end{tabular}
\end{center}


\end{document}
